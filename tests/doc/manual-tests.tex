\documentclass[a4paper]{article}
\usepackage{anysize}
\usepackage{theorem}

\newtheorem{testcase}{Test Case}

\begin{document}

\title{Manual test cases for the installer \\  Revision 2}
\author{Andrew Peace}
\date{}

\maketitle

This document describes manual tests to validate the correctness of
the host installer.  This revision is designed for use against
versions of the installer supporting multiple CDs, driver disks, and
basic Rio to Miami answer reconstruction.

\section{Text user interface -- general}

\begin{testcase}
A welcome screen is displayed that allows the user to cancel their
installation without rebooting their computer.  Cancelling the
installation returns the user to the menu screen.
\end{testcase}

\begin{testcase}
If you select the HTTP or NFS install options, a dialog requiring you
to configure networking in order to proceed is displayed -- otherwise,
it is not displayed.
\end{testcase}

\begin{testcase}
If you saw the networking configuration dialog as a result of choosing
HTTP or NFS as an installation source, then the values you entered should
be presented as the defaults on the network settings screen
(i.e.\ when configuring networking for the host).
\end{testcase}

\begin{testcase}
During an interactive install, the EULA is displayed.
\end{testcase}

\begin{testcase}
The default button for the EULA screen is not Accept.
\end{testcase}

\begin{testcase}
For each screen (listed below for recording purposes), pressing F12
is equivalent to pressing Ok:
\begin{enumerate}
\item Welcome screen;
\item EULA screen;
\item Hardware warnings screen;
\item Installation type screen (upgrade/freshen/clean);
\item Primary disk screen;
\item Guest disks screen;
\item Conflicting volume groups screen;
\item Installation source screen;
\item Networking required screen;
\item HTTP address screen;
\item NFS address screen;
\item Verify source screen;
\item Root password screen;
\item Time-zone region screen;
\item Time-zone city screen;
\item Time configuration screen;
\item NTP servers screen;
\item Network configuration screen;
\item Name service configuration screen;
\item Confirmation screen;
\item Success screen;
\item Insert more media screen;
\item Confirm more media screen.
\end{enumerate}
\end{testcase}

\begin{testcase}
Going backwards in the installer causes information that has already
been entered to be the defaults for the previous screens:
\begin{enumerate}
\item Select installation source;
\item HTTP source;
\item NFS source;
\item Time-zone region;
\item Time-zone city;
\item Time configuration;
\item Select primary disk;
\item Select guest disks;
\item Installation type;
\item Preserve settings;
\item Back-up existing installation;
\item Name service configuration.
\end{enumerate}
\end{testcase}

\begin{testcase}
If the manual time entry option is selected, a dialog is displayed at
the end of the installation that allows the user to enter their
current local time:
\begin{enumerate}
\item the dialog has in it the current local time if the hardware
  clock was set to the current UTC time at boot;
\item the dialog reject invalid dates/times -- we limit the scope of
  this to:
  \begin{enumerate}
  \item fields with values of the incorrect length;
  \item fields with values outside the range of that field (1-31 for
  day, 1-12 for month, 0-23 for hour, 0-59 for minute).
  \end{enumerate}
\end{enumerate}
\end{testcase}

\section{Hardware}

\begin{testcase}
On a system with less than 1GB of RAM, you are warned of this during
installation.
\end{testcase}

\begin{testcase}
On a system without VT support, you are warned of this during
installation.
\end{testcase}

If both of the above are true, the warnings will be on the same
screen.

\begin{testcase}
On a SATA-based system, with the BIOS configured to use AHCI if
appropriate, the installer is able to identify SATA disks attached to
the host.
\end{testcase}

\begin{testcase}
On a system with IDE disks, the installer is able to identify IDE
disks attached to the host.
\end{testcase}

\begin{testcase}
On a system with SCSI disks, the installer is able to identify SCSI
disks attached to the host.
\end{testcase}

\begin{testcase}
On a system with a QLogic HBA, the installer is able to see LUNs
zoned-in to the host.
\end{testcase}

\begin{testcase}
On a system with an Emulex HBA, the installer to able to see LUNs
zoned-in to the host.
\end{testcase}

\begin{testcase}
The installer complete successfully using all default options when
booted with:
\begin{enumerate}
\item the CD in an IDE drive;
\item the CD in a SATA drive;
\item the CD in a USB drive.
\end{enumerate}
\end{testcase}


\section{Conflicting installs}

These tests require a host with two disks.

To set up a system for the following tests, boot into a Linux
environment (this may be done from VT2 on the install CD if you first
set the \texttt{LVM\_SYSTEM\_DIR} environment-variable to be /tmp -- a better
option is James Bulpin's `cleanrd' environment).  Create an LVM
physical volume on both disks. Create a single volume group that spans
both physical volumes.

Ensure there are no previous installations left on the system.  Now
start the installer.

\begin{testcase}
The installer should warn about conflicting LVM volume groups when the
following configurations are chosen:
\begin{enumerate}
\item disk 0 as primary disk, disk 0 for guest storage;
\item disk 0 as primary disk, no guest storage;
\item disk 1 as primary disk, disk 1 for guest storage;
\item disk 1 as primary disk, no guest storage;
\end{enumerate}

Proceeding should cause the volume group you created to be removed in
any of the above scenarios.
\end{testcase}

\begin{testcase}
The installer should NOT warn about conflicting LVM volume groups when
the following configurations are chosen:
\begin{enumerate}
\item disk 0 as primary disk, disk 1 for guest storage;
\item disk 1 as primary disk, disk 0 for guest storage;
\item disk 0 as primary disk, both disks for guest storage;
\item disk 1 as primary disk, both disks for guest storage.
\end{enumerate}
\end{testcase}

\section{Disk options}

These tests are intended to be run on a box without an existing
installation present.

\begin{testcase}
On a single disk host, you are not prompted for disks to install to.
\end{testcase}

\begin{testcase}
On a single disk host, the text for the ``Confirm installation''
screen has appropriate wording.
\end{testcase}

\begin{testcase}
On a multi-disk host, you are prompted for a primary disk to install
to, and a number of disks for guest storage.
\end{testcase}

\begin{testcase}
On a multi-disk host, if you choose no disks for guest storage, a
dialog appears warning you but not preventing you from continuing.
\end{testcase}

\begin{testcase}
On a multi-disk host, the wording for the ``Confirm Installation''
scren is appropriate.
\end{testcase}

The above can be tested without running through the entire installer.

\begin{testcase}
On a multi-disk host, install to the first disk with no SR with the UI.
\end{testcase}

\begin{testcase}
On a multi-disk host, install to the firstdisk with no SR with an
answer-file.
\end{testcase}

First remove all logical volumes and volume groups.  Below is a
suggested answer-file.

\begin{verbatim}
<?xml version="1.0"?>
<installation>
<primary-disk gueststorage="no">sda</primary-disk>
<root-password>xenroot</root-password>
<source type="url">http://www.uk.xensource.com/carbon</source>
<ntp-server>ntp2a.macc.ac.uk</ntp-server>
<interface name="eth0" enabled="yes" proto="dhcp" />
<timezone>Europe/London</timezone>
</installation>
\end{verbatim}

Verify that no volume groups exist still.  Verify that there is no
DEFAULT\_SR value in xensource-inventory, or that the value is the
empty string.

\begin{testcase}
On a multi-disk host, installation to a single disk with an SR with
the UI succeeds.
\end{testcase}

\begin{testcase}
On a multi-disk host, installation to a single disk with an SR with an
answer-file succeeds.
\end{testcase}

First remove all logical volumes and volume groups.  Below is a
suggested answer-file.

\begin{verbatim}
<?xml version="1.0"?>
<installation>
<primary-disk>sda</primary-disk>
<root-password>xenroot</root-password>
<source type="url">http://www.uk.xensource.com/carbon</source>
<ntp-server>ntp2a.macc.ac.uk</ntp-server>
<interface name="eth0" enabled="yes" proto="dhcp" />
<timezone>Europe/London</timezone>
</installation>
\end{verbatim}

Verify that a volume group is created.  Verify that the UUID of the
created volume group is the same as the UUID in the DEFAULT\_SR field
of xensource-inventory.

\begin{testcase}
Installation on both disks of a two-disk host using the UI:
\begin{enumerate}
\item SR on both disks;
\item SR on second disk only.
\end{enumerate}
\end{testcase}

\begin{testcase}
Installation on both disks of a two-disk host using an answerfile:
\begin{enumerate}
\item SR on both disks;
\item SR on second disk only.
\end{enumerate}
\end{testcase}

\section{Installation Sources}

\begin{testcase}
Installation succeeds from the various installation sources using the
UI:
\begin{enumerate}
\item Local CD
\item FTP (anonymous)
\item FTP (user \& password supplied)
\item HTTP
\item HTTP (user \& password supplied)
\item NFS
\end{enumerate}
\end{testcase}

\begin{testcase}
The `Install Linux Pack CD' option has the correct default:
\begin{enumerate}
\item is ticked by default on a machine not supporting HVM/AMD-V;
\item is not ticked by default on a machine that does support HMV/AMD-V.
\end{enumerate}
\end{testcase}

\begin{testcase}
When installing on a host without hardware assist support, unticking
the `Install Linux Pack CD' box causes a warning to be displayed on
the next screen.
\end{testcase}

\begin{testcase}
When installing from local media, an error appears when leaving the
``Select Installation Source'' screen if the first (Base Pack) CD is
not present.
\end{testcase}

\begin{testcase}
Choosing to install the Linux pack has the expected effect of
requesting the new CD, and installing the Linux XGTs, etc.
\end{testcase}

\begin{testcase}
When installing the Linux Pack:
\begin{enumerate}
\item the user is appropriately prompted partway through installation
to swap discs.
\item the user is only prompted once;
\item the contents of the media are shown;
\item the verify function works;
\item the user can cancel the request for further media, and their
  installation completes successfully with only the base functionality
  present;
\item re-inserting the first disc again shows a screen indicating it
  is alredy installed.
\end{enumerate}
\end{testcase}

\begin{testcase}
Installation proceeds from the specified media when running using an
answer-file
\begin{enumerate}
\item FTP (anonymous)
\item FTP (user \& password supplied)
\item HTTP
\item HTTP (user \& password supplied)
\item NFS
\end{enumerate}
\end{testcase}
Note that you can specify these by using 'url', and 'nfs' as
the 'source-media' value in your answer-file.

\begin{testcase}
Sources are correctly identified when using a remote installation
repository:
\begin{enumerate}
\item The installer will give an error and allow the user to re-enter
  their source address if it is given a faulty source address:
  \begin{enumerate}
  \item the address exists but is not a repository;
  \item the address does not exist;
  \item the address is invalid.
  \end{enumerate}
\end{enumerate}
\end{testcase}

\begin{testcase}
If the CD is not present when the install begins, the installer will
bring up a user-friendly dialog asking for it to be re-inserted, with
Retry and Cancel options.
\end{testcase}

\begin{testcase}
When the source verification option is used, an error dialog that does
not force the user to restart their computer, but does not allow them
to proceed using the source they specified, is displayed when any of
the following is true:
\begin{enumerate}
\item The .md5 file for a package listed in the PACKAGES file is
  missing.
\item The .tar.bz2 file for a package listed in the PACKAGES file is
  missing.
\item The .md5 file has a different md5sum recorded than the actual
  md5sum of the .tar.bz2 file for one or more packages listed in the
  PACKAGES file.
\end{enumerate}
\end{testcase}

\begin{testcase}
When a package is found to be faulty by the source verification
option:
\begin{enumerate}
\item It is explicitly named in the error dialog;
\item Only faulty packages are listed in this dialog;
\item The package is listed without Python class names or syntax
(i.e.\ in a sentential form).
\end{enumerate}
\end{testcase}

\section{Re-install over existing product}

These tests are applicable only to interactive installs.

This section requires you to have already installed an existing
product.  We expect that Miami versions up to GA will be able to
upgrade versions labelled 4.0.1 and higher.  It is recommended that
this set of tests should use Miami RC1 as the upgrade source version.

\begin{testcase}
On a host with Rio GA installed the user may choose to upgrade it.
\end{testcase}

\begin{testcase}
The option to back up an existing installation is available:
\begin{enumerate}
\item the correct back-up partition is identified in the dialog text
  (see below);
\item is is not displayed if the clean install option was selected.
\end{enumerate}
\end{testcase}
(The backup partition should be the second partition of the disk that
the product to be overwritten is installed on.)

\begin{testcase}
If selected, the back-up function should cause the original
installation to be backed up to the backup partition.
\end{testcase}

\begin{testcase}
The screens used to enter configuation settings are displayed only as
appropriate:
\begin{enumerate}
\item they are not displayed if an upgrade or freshen option was selected;
\item they are displayed if the clean install option was selected.
\end{enumerate}
\end{testcase}


\section{Module loading}

\begin{testcase}
The ``extramodules'' kernel command-line option works correctly:
\begin{enumerate}
\item extramodules=xfs,vfat causes these modules to be loaded by the
  installer and added to the dom0 initrd;
\item extramodules=vfat cause the vfat module to be loaded by the
  installer and added to the dom0 initrd.
\end{enumerate}
\end{testcase}

\begin{testcase}
The ``blacklist'' kernel command-line option works correctly:
\begin{enumerate}
\item blacklist=ide-generic,sr-mod prevents these modules from being
  loaded by the installer and added to the dom0 initrd;
\item blacklist=sr-mod prevents the sr-mode module from begin loaded
  by the installer and added to the dom0 initrd.
\end{enumerate}
This test should also be performed with a module that is known to be
detected as needed on the hardware you are testing with.  USB modules
are currently an exception in this test case.
\end{testcase}

\begin{testcase} \label{tc:interactive-modules}
The ``interactive-modules'' kernel command-line option causes the user
to be prompted as to whether each proposed module should be loaded.
\end{testcase}

\section{Installation over serial}

For these tests to be valid, you will need to configure the installer
with the ``output=ttyS0'' kernel command-line option.  These tests are
easiest to perform using PXE.

\begin{testcase}
When running with the ``output=ttyS0'' option the hardware detection
progress bar appears on the serial console at boot.
\end{testcase}

\begin{testcase}
Test case \ref{tc:interactive-modules} still passes when running on serial.
\end{testcase}

\begin{testcase}
The menu is displayed on both consoles even when ``output=ttyS0'' is
put onto the command-line.  (This is only necessary for
troubleshooting options to work correctly.)
\end{testcase}

\begin{testcase}
If the installer is started on the serial console, it cannot be
started from VT1; a friendly error message should be displayed.
\end{testcase}

\section{Driver disks}

\begin{testcase}
The option to use a driver disk is present on the initial menu.
\end{testcase}

\begin{testcase}
When the user chooses to use a driver disk, they are asked the source
to be used.
\end{testcase}

\section{Regression avoidance tests}

\begin{testcase}
Pressing Ctrl+C on the VT where installer logging is displayed does
not crash the installer
\end{testcase}

\end{document}
